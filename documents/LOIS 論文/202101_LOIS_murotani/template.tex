%%「論文」,「レター」,「レター(C分冊)」,「技術研究報告」などのテンプレート
%% v3.3 [2020/06/02]


%% 4. 「技術研究報告」
\documentclass[technicalreport,dvipdfmx]{ieicej}
%\usepackage[dvips]{graphicx}
%\usepackage[dvipdfmx]{graphicx,xcolor}
\usepackage[fleqn]{amsmath}
\usepackage{newtxtext}% 英数字フォントの設定を変更しないでください
\usepackage[varg]{newtxmath}% % 英数字フォントの設定を変更しないでください
\usepackage{latexsym}
%\usepackage{amssymb}
\usepackage[dvipdfmx]{graphicx}

\jtitle{}
\jsubtitle{}
\etitle{}
\esubtitle{}
\authorlist{%
 \authorentry[]{}{}{}
% \authorentry[メールアドレス]{和文著者名}{英文著者名}{所属ラベル}
}
\affiliate[]{}{}
%\affiliate[所属ラベル]{和文勤務先\\ 連絡先住所}{英文勤務先\\ 英文連絡先住所}
\jalcdoi{???????????}% ← このままにしておいてください

\begin{document}
\begin{jabstract}
%和文あらまし
\end{jabstract}
\begin{jkeyword}
%和文キーワード
\end{jkeyword}
\begin{eabstract}
%英文アブストラクト
\end{eabstract}
\begin{ekeyword}
%英文キーワード
\end{ekeyword}
\maketitle

\section{はじめに}

\section{準備}

\section{提案手法}
\subsection{システム概要}

\subsection{ユースケース}

\subsection{システムアーキテクチャ}

\subsection{ドメインモデリング}
ShelterNaviにおけるドメインモデル図\ref{fig:domain}にを示す.
ShelterNaviには,図\ref{fig:domain}で表されている5つのドメインが存在し,それぞれShelter(避難所),User(住民・ユーザ),Check-In(避難所へのチェックインまたはチェックアウト),ShelterState(避難所の状態),UserState(ユーザの状態)となっている.Check-Inドメインを通じてユーザとそのユーザが利用した避難所を関連付けさせるために,Check-Inドメインは必ず1つのUserドメインと1つのShelterドメインと結びつく.また,ShelterStateドメインは各避難所の状態を表しており,UserStateドメインはユーザの避難状態を示している.
以下では,ShelterNaviの各ドメインを構成する要素について説明する.Shelterドメインにおいては,システムがどの避難所か特定するためのIDが必要であり,避難するユーザに向けて避難所の名称,位置情報を示す必要がある.そして,避難所の開放情報も取り扱う.以上のことからShelterドメインでは以下のスキーマを規定する.

\begin{itemize}
    \item{\textbf{sid:}}避難所ID
    \item{\textbf{name:}}避難所の名称
    \item{\textbf{address:}}避難所の住所
    \item{\textbf{lat:}}避難所の緯度
    \item{\textbf{lng:}}避難所の経度
    \item{\textbf{capacity:}}避難所の収容可能人数
    \item{\textbf{isActive:}}避難所が利用可能か
\end{itemize}

続いてUserドメインにおいては,ユーザ情報を一意に取り扱うためのユーザIDが必要である.また,本サービスに登録するためのemailアドレスとパスワード,(名前,住所:必要性をどう示すか),そして混雑度を計算するための世帯人数を取り扱う.% 本アプリにおける名前,住所,電話番号の必要性
以上のことからUserドメインでは以下のスキーマを規定する.

\begin{itemize}
    \item{\textbf{uid:}}ユーザID
    \item{\textbf{email:}}メールアドレス
    \item{\textbf{password:}}パスワード
    \item{\textbf{name:}}ユーザの名前
    \item{\textbf{address:}}ユーザの住所
    \item{\textbf{of families:}}世帯人数
\end{itemize}

Check-Inドメインにおいては,どのユーザがどの避難所を利用しているかを管理するためにユーザIDと避難所IDが必要である.また,複数の避難所への多重チェックインがないかを確認するために,チェックイン時刻と,チェックアウト時刻も取り扱う.以上のことからCheck-Inドメインでは以下のスキーマを規定する.

\begin{itemize}
    \item{\textbf{cid:}}チェックインID
    \item{\textbf{uid:}}ユーザID
    \item{\textbf{sid:}}避難所ID
    \item{\textbf{checkin-datetime:}}チェックイン時刻
    \item{\textbf{checkout-datetime:}}チェックアウト時刻
\end{itemize}

ShelterStateドメインにおいては,コロナ禍における避難所での密を考慮すべく,避難所の混雑度も取り扱う.この混雑度は,避難所の収容可能人数に対しての現在の収容人数の割合から求められるので収容人数も管理する.そしてチェックイン,チェックアウトによる収容人数の変化を見るために,避難所の状態の最終更新日時も取り扱う.以上のことからShelterStateドメインでは以下のスキーマを規定する.

% 現在の収容人数のスキーマ名なににするべきか...
\begin{itemize}
     \item{\textbf{id:}}避難所の状態ID
     \item{\textbf{sid:}}避難所ID
     \item{\textbf{num of accomodated:}}収容人数
     \item{\textbf{congestion:}}混雑度
     \item{\textbf{changedAt:}}更新日時
 \end{itemize}

UserStateドメインでは,ユーザの避難状態の履歴を管理する.特定のユーザの履歴を見るためにユーザIDが必要である.そして,準備中,避難済み等の避難状態を表すスキーマを取り扱い,避難状態が変化した際の日時を記録する.なお,このUserStateドメインでは過去のデータを参照できるようにしておくために,1つのデータを更新していくのではなく,新規データを追加していく形をとる.

 \begin{itemize}
     \item{\textbf{id:}}ユーザの状態ID
     \item{\textbf{uid:}}ユーザID
     \item{\textbf{state:}}避難状態
     \item{\textbf{changedAt:}}更新日時
 \end{itemize}

\begin{figure}[htbp]
     \begin{center}
          \includegraphics[scale=0.5,pagebox=cropbox,clip]{domain_model.pdf}
          \caption{ドメインモデル図}
          \label{fig:domain}
     \end{center}
\end{figure}
% CitizenStateドメインではユーザの状態しか取り扱わない?
% Citizenドメインのフィールドとするのはダメなのかチェックする

\subsection{主要なサービス}
ShelterNaviでは他のアプリケーションとの連携や拡張性を考慮し,HTTPを介して外部から利用できるAPIを配備した.以下にAPIの詳細を示す.

\subsubsection{シェルターサービス}
\begin{itemize}
    \item{\textbf{createShelter( shelterForm ):}
         避難所ID,避難所名,位置情報を基に避難所データを作成し,取得する.}
    \item{\textbf{getShelter( sid ):}
         避難所IDを指定することで該当する避難所データを取得する.}
    \item{\textbf{deleteShelter( sid ):}
         避難所IDを指定することで該当する避難所データを削除する.}
    \item{\textbf{clearAllShelters():}
         全ての避難所データを削除する.}
    \item{\textbf{getAllShelters():}
         全ての避難所データを取得する.}
    \item{\textbf{searchSheltersByDistance( lng, lat, distance ):}
         経度,緯度,そして距離を指定することで,指定位置座標(lng, lat)から半径distance$\rm[km]$以内にある避難所データ全てを取得する.当APIでは,地球上における大圏(大円)距離を計算する手法を使用しており,地球の半径を6371[km],ユーザの経度座標をlng,緯度座標をlat,各避難所データの経度座標をs\_lng,緯度座標をs\_latとし,ユーザから半径distance[km]以内に存在する避難所を検索するものとして以下の計算式を用いる.

         \begin{eqnarray*}
             6371 \arccos ( \cos( radians( lat ) ) \times \cos( radians( s\_lat ) ) \\
             \times \cos( radians( s\_lng ) - radians( lng ) ) + \sin( radians( lat ) ) \\
             \times \sin( radians( s\_lat ) ) ) \leqq distance
         \end{eqnarray*}}
    \item{\textbf{searchSheltersByKeyword( keyword ):}
         文字列を指定することで,全避難所データの避難所名,または避難所の住所に部分一致するものがないか検索し,該当するデータを全て取得する.}
\end{itemize}

\subsubsection{ユーザサービス}
\begin{itemize}
    \item{\textbf{createUser( userForm ):}
         メールアドレス,パスワード,世帯人数等のユーザ情報を元に新規アカウントを作成し,取得する.}
    \item{\textbf{getUser( uid ):}
         ユーザIDを指定することで該当するユーザアカウントを取得する.}
    \item{\textbf{deleteUser( uid ):}
         ユーザIDを指定することで該当するユーザアカウントを削除する.}
\end{itemize}

\subsubsection{チェックインサービス}
\begin{itemize}
    \item{\textbf{checkIn( uid, sid ):}
         ユーザIDと避難所IDを指定することで,チェックインデータを作成し,更新日時を記録する.}
    \item{\textbf{checkOut( uid, sid ):}
         ユーザIDと避難所IDから一意に定まるチェックインデータを取得し,チェックアウト時刻を更新する.}
\end{itemize}

\subsubsection{シェルターステートサービス}
\begin{itemize}
     \item{\textbf{checkIn( uid, sid ):}
          ユーザIDと避難所IDを指定することで,避難所の収容人数をユーザの世帯人数分増加させ,避難所の混雑度を計算し,更新する.また,更新日時を記録する.}
     \item{\textbf{checkOut( uid, sid ):}
          ユーザIDと避難所IDを指定することで,避難所の収容人数をユーザの世帯人数分減少させ,避難所の混雑度を計算し,更新する.また,更新日時を記録する.}
\end{itemize}

\subsubsection{ユーザステートサービス}
\begin{itemize}
     \item{\textbf{checkIn( uid ):}
          ユーザIDを指定することで,特定のユーザの避難状態を「避難済み」にしたUserStateのデータを新規で作成する.また,更新日時も併せて記録する.}
     \item{\textbf{checkOut( uid ):}
          ユーザIDを指定することで,特定のユーザの避難状態を「準備中」にしたUserStateのデータを新規で作成する.また,更新日時も併せて記録する.}
\end{itemize}

\section{実装}
\subsection{ShelterNaviプロトタイプの実装}
今回は以下の開発環境で「ユーザサービス」,「シェルターサービス」,また「チェックインサービス」の一部の開発を行った.
\begin{itemize}
    \item サーバ開発言語:Java
    \item クライアント開発言語:HTML5,JavaScript
    \item CSSライブラリ:BootStrap
    \item データベース:MySQL 8.0.20
    \item Webサーバ:Apache Tomcat
    \item Webサービスフレームワーク:SpringBoot(Java)
\end{itemize}
以下ではShelterNaviの「サインアップ」,「ログイン」,「避難所検索」,「チェックイン」の各画面構成に併せて実装の詳細を述べる.

\subsection{サインアップ}
サインアップ画面を図\ref{fig:signup}に示す.サインアップ画面は,ログイン時に必要なメールアドレスとパスワードからなる「ログイン情報」とログイン後,システム内で扱われる名前,住所,世帯人数,電話番号(任意)からなる「個人情報」の2つの要素で構成されている.これらの情報を入力したうえで図\ref{fig:signup}下部の登録ボタンをクリックすれば,ユーザ新規登録の完了となる.

\subsection{ログイン}
%今回の実装においては,セキュリティ性を担保するためにJavaのフレームワークであるSpring Securityを用いる.このフレームワークの機能を利用すれば,指定したURL内で,IDとパスワードをPostする特定のAPIを利用することで認証が可能になる.また,ユーザオブジェクトに権限を付与し,その権限に応じた認可を与えることも可能になる.本アプリケーションでは,通常の工程でユーザを作成した場合,CITIZEN(住民・一般ユーザ)の役割が付与される.実際には,ログイン機能における認証時にこの役割を見ることで,セッションに対して役割に対応した権限を付与する.これによりログイン後のユーザに対する各ページへの認可が可能になる.

\subsection{避難所検索}
%実際は管理者が直接データを入れることになりそうなので機能ではない?

\subsection{チェックイン}
%本アプリケーションでは,避難所を可視化する上でGoogle Mapを利用している.避難所を取得するAPIで避難所データを取得し,それらのデータをGoogleMapsAPIで利用することで,地図上への避難所の可視化を行っている.また,ShelterNaviではユーザの現在位置に応じて地図上に表示する避難所を変更しており,これは「3.5 主要なサービス」で言及したsearchSheltersByDistance(lng, lat, distance)をを利用している.

\begin{figure}[htbp]
     \begin{center}
          \includegraphics[scale=0.5,pagebox=cropbox,clip]{signup.png}
          \caption{新規登録画面}
          \label{fig:signup}
     \end{center}
\end{figure}

\begin{figure}[htbp]
     \begin{center}
          \includegraphics[scale=0.5,pagebox=cropbox,clip]{login.png}
          \caption{ログイン画面}
          \label{fig:login}
     \end{center}
\end{figure}

\begin{figure}[htbp]
     \begin{center}
          \includegraphics[scale=0.6,pagebox=cropbox,clip]{search_shelter.png}
          \caption{避難所検索画面}
          \label{fig:search_shelter}
     \end{center}
\end{figure}

\section{考察・評価}

\section{おわりに}


%\bibliographystyle{sieicej}
%\bibliography{myrefs}
\begin{thebibliography}{99}% 文献数が10未満の時 {9}
\bibitem{}
\end{thebibliography}

\end{document}


%% 1. 「論文」
\documentclass[paper]{ieicej}
%\documentclass[invited]{ieicej}% 招待論文
%\documentclass[survey]{ieicej}% サーベイ論文
%\documentclass[comment]{ieicej}% 解説論文
%\usepackage[dvips]{graphicx}
%\usepackage[dvipdfmx]{graphicx,xcolor}
\usepackage[fleqn]{amsmath}
\usepackage{newtxtext}% 英数字フォントの設定を変更しないでください
\usepackage[varg]{newtxmath}% % 英数字フォントの設定を変更しないでください
\usepackage{latexsym}
%\usepackage{amssymb}

\setcounter{page}{1}

\field{}
\jtitle{}
\etitle{}
\authorlist{%
 \authorentry{}{}{}\MembershipNumber{}
 %\authorentry{和文著者名}{英文著者名}{所属ラベル}\MembershipNumber{}
 %\authorentry[メールアドレス]{和文著者名}{英文著者名}{所属ラベル}\MembershipNumber{}
 %\authorentry{和文著者名}{英文著者名}{所属ラベル}[現在の所属ラベル]\MembershipNumber{}
}
\affiliate[]{}{}
%\affiliate[所属ラベル]{和文所属}{英文所属}
%\paffiliate[]{}
%\paffiliate[現在の所属ラベル]{和文所属}
\jalcdoi{???????????}% ← このままにしておいてください

\begin{document}
\begin{abstract}
%和文あらまし 500字以内
\end{abstract}
\begin{keyword}
%和文キーワード 4〜5語
\end{keyword}
\begin{eabstract}
%英文アブストラクト 100 words
\end{eabstract}
\begin{ekeyword}
%英文キーワード
\end{ekeyword}
\maketitle



\ack %% 謝辞

%\bibliographystyle{sieicej}
%\bibliography{myrefs}
\begin{thebibliography}{99}% 文献数が10未満の時 {9}
\bibitem{}
\end{thebibliography}

\appendix

\begin{biography}
\profile{}{}{}
%\profile{会員種別}{名前}{紹介文}% 顔写真あり
%\profile*{会員種別}{名前}{紹介文}% 顔写真なし
\end{biography}

\end{document}



%% 2. 「レター」
\documentclass[letter]{ieicej}
%\usepackage[dvips]{graphicx}
%\usepackage[dvipdfmx]{graphicx,xcolor}
\usepackage[fleqn]{amsmath}
\usepackage{newtxtext}% 英数字フォントの設定を変更しないでください
\usepackage[varg]{newtxmath}% % 英数字フォントの設定を変更しないでください
\usepackage{latexsym}
%\usepackage{amssymb}

\setcounter{page}{1}

\typeofletter{研究速報}
%\typeofletter{紙上討論}
%\typeofletter{問題提起}
%\typeofletter{ショートノート}
\field{}
\jtitle{}
\etitle{}
\authorlist{%
 \authorentry{}{}{}{}\MembershipNumber{}
 %\authorentry{和文著者名}{英文著者名}{会員種別}{所属ラベル}\MembershipNumber{}
 %\authorentry{和文著者名}{英文著者名}{会員種別}{所属ラベル}[現在の所属ラベル]\MembershipNumber{}
}
\affiliate[]{}{}
%\affiliate[所属ラベル]{和文所属}{英文所属}
%\paffiliate[]{}
%\paffiliate[現在の所属ラベル]{和文所属}
\jalcdoi{???????????}% ← このままにしておいてください

\begin{document}
\maketitle
\begin{abstract}
%和文あらまし 120字以内
\end{abstract}
\begin{keyword}
%和文キーワード 4〜5語
\end{keyword}
\begin{eabstract}
%英文アブストラクト 50 words
\end{eabstract}
\begin{ekeyword}
%英文キーワード
\end{ekeyword}

\section{まえがき}


\ack %% 謝辞

%\bibliographystyle{sieicej}
%\bibliography{myrefs}
\begin{thebibliography}{99}% 文献数が10未満の時 {9}
\bibitem{}
\end{thebibliography}

\appendix
\section{}

\end{document}


%% 3. 「レター(C分冊)」
\documentclass[electronicsletter]{ieicej}
%\usepackage[dvips]{graphicx}
%\usepackage[dvipdfmx]{graphicx,xcolor}
\usepackage[fleqn]{amsmath}
\usepackage{newtxtext}% 英数字フォントの設定を変更しないでください
\usepackage[varg]{newtxmath}% % 英数字フォントの設定を変更しないでください
\usepackage{latexsym}
%\usepackage{amssymb}

\setcounter{page}{1}

\field{}
\jtitle{}
\etitle{}
\authorlist{%
 \authorentry{}{}{}{}\MembershipNumber{}
 %\authorentry{和文著者名}{英文著者名}{会員種別}{所属ラベル}\MembershipNumber{}
 %\authorentry{和文著者名}{英文著者名}{会員種別}{所属ラベル}[現在の所属ラベル]\MembershipNumber{}
}
\affiliate[]{}{}
%\affiliate[所属ラベル]{和文所属}{英文所属}
%\paffiliate[]{}
%\paffiliate[現在の所属ラベル]{和文所属}
\jalcdoi{???????????}% ← このままにしておいてください

\begin{document}
\begin{abstract}
%和文あらまし 120字以内
\end{abstract}
\begin{keyword}
%和文キーワード 4〜5語
\end{keyword}
\begin{eabstract}
%英文アブストラクト 50 words
\end{eabstract}
\begin{ekeyword}
%英文キーワード
\end{ekeyword}
\maketitle

\section{まえがき}


\ack %% 謝辞

%\bibliographystyle{sieicej}
%\bibliography{myrefs}
\begin{thebibliography}{99}% 文献数が 10 未満の時 {9}
\bibitem{}
\end{thebibliography}

\appendix
\section{}

\end{document}
